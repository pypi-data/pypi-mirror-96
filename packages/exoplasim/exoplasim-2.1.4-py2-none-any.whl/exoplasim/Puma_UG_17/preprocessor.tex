In many cases the setup of PUMA experiments can be defined
using namelist variables either via MoSt or with editing
the namelist file. In these cases PUMA can run without
any startup files containing boundary conditions. \\

For more complex experiments, like changes in orography
or ground temperature, predefined vertical and horizontal
gradients of the restoration temperature field
and more, it is necessary to create files for boundary
conditions. \\

This is done with the PPP (short for Puma Pre-Processor).

The PPP is a stand alone program, that can be called
inside the modelstarter MoSt or explicitely by the user.
It shares the namelist file {\bf puma\_namelist} with
PUMA, because both programs must use the same parameters
for consistency.

The use in MoSt is currently restricted for using an orography
in PUMA. If the orography option is checked in MoSt the 
PPP will be run before creating the run time environment for
the model. The PPP creates startup definitions for 
orography, constant and time variable part of the restoration
temperature and an initial field for surface pressure.

Additionally the simple orography modifier of MoSt may be used
to rise or lower parts of the orography. A mouseclick on the
button {\bf Preprocess} will then call the PPP and make
all necessary adjustions to start fields. \\
More complex setups must be performed by either using some of
the PPP namelist parameters or by adding code to PPP itself.
This requires however a good knowledge of the FORTRAN-90
programming language and of the model interna.
The source code is in the file Most16/puma/src/ppp.f90.
To make changes easier the PPP has two subroutines
named {\bf modify\_orography} and {\bf modify\_ground\_temperature}.
These are the recommended places to add user code. \\

More details can be found in the FORTRAN-90 code of the PPP itself.

