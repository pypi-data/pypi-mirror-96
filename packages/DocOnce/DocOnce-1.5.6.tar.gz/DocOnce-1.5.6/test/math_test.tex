\documentclass[]{article}
\usepackage{lmodern}
\usepackage{amssymb,amsmath}
\usepackage{ifxetex,ifluatex}
\usepackage{fixltx2e} % provides \textsubscript
\ifnum 0\ifxetex 1\fi\ifluatex 1\fi=0 % if pdftex
  \usepackage[T1]{fontenc}
  \usepackage[utf8]{inputenc}
\else % if luatex or xelatex
  \ifxetex
    \usepackage{mathspec}
  \else
    \usepackage{fontspec}
  \fi
  \defaultfontfeatures{Ligatures=TeX,Scale=MatchLowercase}
\fi
% use upquote if available, for straight quotes in verbatim environments
\IfFileExists{upquote.sty}{\usepackage{upquote}}{}
% use microtype if available
\IfFileExists{microtype.sty}{%
\usepackage[]{microtype}
\UseMicrotypeSet[protrusion]{basicmath} % disable protrusion for tt fonts
}{}
\PassOptionsToPackage{hyphens}{url} % url is loaded by hyperref
\usepackage[unicode=true]{hyperref}
\hypersetup{
            pdftitle={How various formats can deal with LaTeX math},
            pdfauthor={Hans Petter Langtangen at Simula Research Laboratory and University of Oslo},
            pdfborder={0 0 0},
            breaklinks=true}
\urlstyle{same}  % don't use monospace font for urls
\IfFileExists{parskip.sty}{%
\usepackage{parskip}
}{% else
\setlength{\parindent}{0pt}
\setlength{\parskip}{6pt plus 2pt minus 1pt}
}
\setlength{\emergencystretch}{3em}  % prevent overfull lines
\providecommand{\tightlist}{%
  \setlength{\itemsep}{0pt}\setlength{\parskip}{0pt}}
\setcounter{secnumdepth}{0}
% Redefines (sub)paragraphs to behave more like sections
\ifx\paragraph\undefined\else
\let\oldparagraph\paragraph
\renewcommand{\paragraph}[1]{\oldparagraph{#1}\mbox{}}
\fi
\ifx\subparagraph\undefined\else
\let\oldsubparagraph\subparagraph
\renewcommand{\subparagraph}[1]{\oldsubparagraph{#1}\mbox{}}
\fi

% set default figure placement to htbp
\makeatletter
\def\fps@figure{htbp}
\makeatother


\title{How various formats can deal with LaTeX math}
\author{\textbf{Hans Petter Langtangen} at Simula Research Laboratory and
University of Oslo}
\date{Mar 2, 2021}

\begin{document}
\maketitle

\emph{Summary.} The purpose of this document is to test LaTeX math in
DocOnce with various output formats. Most LaTeX math constructions are
renedered correctly by MathJax in plain HTML, but some combinations of
constructions may fail. Unfortunately, only a subset of what works in
html MathJax also works in sphinx MathJax. The same is true for markdown
MathJax expresions (e.g., Jupyter notebooks). Tests and examples are
provided to illustrate what may go wrong.

The recommendation for writing math that translates to MathJax in html,
sphinx, and markdown is to stick to the environments
\texttt{\textbackslash{}{[}\ ...\ \textbackslash{}{]}},
\texttt{equation}, \texttt{equation*}, \texttt{align}, \texttt{align*},
\texttt{alignat}, and \texttt{alignat*} only. Test the math with sphinx
output; if it works in that format, it should work elsewhere too.

The current version of the document is translated from DocOnce source to
the format \textbf{pandoc}.

\subsection{Test of equation
environments}\label{test-of-equation-environments}

\subsubsection{Test 1: Inline math}\label{test-1-inline-math}

We can get an inline equation \texttt{\$u(t)=e\^{}\{-at\}\$} rendered as
\(u(t)=e^{-at}\).

\subsubsection{Test 2: A single equation with
label}\label{test-2-a-single-equation-with-label}

An equation with number,

\begin{verbatim}
!bt
\begin{equation} u(t)=e^{-at} \label{eq1a}\end{equation}
!et
\end{verbatim}

looks like

\[
\begin{equation} u(t)=e^{-at} \label{_eq1a}\end{equation}
\] Maybe this multi-line version is what we actually prefer to write:

\begin{verbatim}
!bt
\begin{equation}
u(t)=e^{-at}
\label{eq1b}
\end{equation}
!et
\end{verbatim}

The result is the same:

\[
\begin{equation}
u(t)=e^{-at} \label{_eq1b}
\end{equation}
\] We can refer to this equation through its label \texttt{eq1b}:
(\protect\hyperlink{_eq1b}{\_eq1b}).

\subsubsection{Test 3: Multiple, aligned equations without label and
number}\label{test-3-multiple-aligned-equations-without-label-and-number}

MathJax has historically had some problems with rendering many LaTeX
math environments, but the \texttt{align*} and \texttt{align}
environments have always worked.

\begin{verbatim}
!bt
\begin{align*}
u(t)&=e^{-at}\\ 
v(t) - 1 &= \frac{du}{dt}
\end{align*}
!et
\end{verbatim}

Result:

\[
\begin{align*}
u(t)&=e^{-at}\\ 
v(t) - 1 &= \frac{du}{dt}
\end{align*}
\]

\subsubsection{Test 4: Multiple, aligned equations with
label}\label{test-4-multiple-aligned-equations-with-label}

Here, we use \texttt{align} with user-prescribed labels:

\begin{verbatim}
!bt
\begin{align}
u(t)&=e^{-at}
\label{eq2b}\\ 
v(t) - 1 &= \frac{du}{dt}
\label{eq3b}
\end{align}
!et
\end{verbatim}

Result:

\[
\begin{align}
u(t)&=e^{-at}
\label{_eq2b}\\ 
v(t) - 1 &= \frac{du}{dt}
\label{_eq3b}
\end{align}
\] We can refer to the last equations as the system
(\protect\hyperlink{_eq2b}{\_eq2b})-(\protect\hyperlink{_eq3b}{\_eq3b}).

\subsubsection{Test 5: Multiple, aligned equations without
label}\label{test-5-multiple-aligned-equations-without-label}

In LaTeX, equations within an \texttt{align} environment is
automatically given numbers. To ensure that an html document with
MathJax gets the same equation numbers as its latex/pdflatex companion,
DocOnce generates labels in equations where there is no label
prescribed. For example,

\begin{verbatim}
!bt
\begin{align}
u(t)&=e^{-at}
\\ 
v(t) - 1 &= \frac{du}{dt}
\end{align}
!et
\end{verbatim}

is edited to something like

\begin{verbatim}
!bt
\begin{align}
u(t)&=e^{-at}
\label{_auto5}\\ 
v(t) - 1 &= \frac{du}{dt}
\label{_auto6}
\end{align}
!et
\end{verbatim}

and the output gets the two equation numbered.

\[
\begin{align}
u(t)&=e^{-at}\\ 
v(t) - 1 &= \frac{du}{dt}
\end{align}
\]

\subsubsection{Test 6: Multiple, aligned equations with multiple
alignments}\label{test-6-multiple-aligned-equations-with-multiple-alignments}

The \texttt{align} environment can be used with two \texttt{\&}
alignment characters, e.g.,

\begin{verbatim}
!bt
\begin{align}
\frac{\partial u}{\partial t} &= \nabla^2 u, & x\in (0,L),
\ t\in (0,T]\\ 
u(0,t) &= u_0(x), & x\in [0,L]
\end{align}
!et
\end{verbatim}

The result in pandoc becomes

\[
\begin{align}
\frac{\partial u}{\partial t} &= \nabla^2 u, & x\in (0,L),
\ t\in (0,T]\\ 
u(0,t) &= u_0(x), & x\in [0,L]
\end{align}
\]

A better solution is usually to use an \texttt{alignat} environment:

\begin{verbatim}
!bt
\begin{alignat}{2}
\frac{\partial u}{\partial t} &= \nabla^2 u, & x\in (0,L),
\ t\in (0,T]\\ 
u(0,t) &= u_0(x), & x\in [0,L]
\end{alignat}
!et
\end{verbatim}

with the rendered result

\[
\begin{alignat}{2}
\frac{\partial u}{\partial t} &= \nabla^2 u, & x\in (0,L),
\ t\in (0,T]\\ 
u(0,t) &= u_0(x), & x\in [0,L]
\end{alignat}
\]

If DocOnce had not rewritten the above equations, they would be rendered
in pandoc as

\[
\begin{alignat}{2}
\frac{\partial u}{\partial t} &= \nabla^2 u, & x\in (0,L),
\ t\in (0,T]\\ 
u(0,t) &= u_0(x), & x\in [0,L]
\end{alignat}
\]

\subsubsection{Test 7: Multiple, aligned eqnarray equations without
label}\label{test-7-multiple-aligned-eqnarray-equations-without-label}

Let us try the old \texttt{eqnarray*} environment.

\begin{verbatim}
!bt
\begin{eqnarray*}
u(t)&=& e^{-at}\\ 
v(t) - 1 &=& \frac{du}{dt}
\end{eqnarray*}
!et
\end{verbatim}

which results in

\[
\begin{eqnarray*}
u(t)&=& e^{-at}\\ 
v(t) - 1 &=& \frac{du}{dt}
\end{eqnarray*}
\]

\subsubsection{Test 8: Multiple, eqnarrayed equations with
label}\label{test-8-multiple-eqnarrayed-equations-with-label}

Here we use \texttt{eqnarray} with labels:

\begin{verbatim}
!bt
\begin{eqnarray}
u(t)&=& e^{-at}
\label{eq2c}\\ 
v(t) - 1 &=& \frac{du}{dt}
\label{eq3c}
\end{eqnarray}
!et
\end{verbatim}

which results in

\[
\begin{eqnarray}
u(t)&=& e^{-at} \label{_eq2c}\\ 
v(t) - 1 &=& \frac{du}{dt} \label{_eq3c}
\end{eqnarray}
\] Can we refer to the last equations as the system
(\protect\hyperlink{_eq2c}{\_eq2c})-(\protect\hyperlink{_eq3c}{\_eq3c})
in the pandoc format?

\subsubsection{\texorpdfstring{Test 9: The \texttt{multiline}
environment with label and
number}{Test 9: The multiline environment with label and number}}\label{test-9-the-multiline-environment-with-label-and-number}

The LaTeX code

\begin{verbatim}
!bt
\begin{multline}
\int_a^b f(x)dx = \sum_{j=0}^{n} \frac{1}{2} h(f(a+jh) +
f(a+(j+1)h)) \\ 
=\frac{h}{2}f(a) + \frac{h}{2}f(b) + \sum_{j=1}^n f(a+jh)
\label{multiline:eq1}
\end{multline}
!et
\end{verbatim}

gets rendered as

\[
\begin{multline}
\int_a^b f(x)dx = \sum_{j=0}^{n} \frac{1}{2} h(f(a+jh) +
f(a+(j+1)h)) \\ 
=\frac{h}{2}f(a) + \frac{h}{2}f(b) + \sum_{j=1}^n f(a+jh)
\label{_multiline:eq1}
\end{multline}
\] and we can hopefully refer to the Trapezoidal rule as the formula
(\protect\hyperlink{_multiline:eq1}{\_multiline:eq1}).

\subsubsection{Test 10: Splitting equations using a split
environment}\label{test-10-splitting-equations-using-a-split-environment}

Although \texttt{align} can be used to split too long equations, a more
obvious command is \texttt{split}:

\begin{verbatim}
!bt
\begin{equation}
\begin{split}
\int_a^b f(x)dx = \sum_{j=0}^{n} \frac{1}{2} h(f(a+jh) +
f(a+(j+1)h)) \\ 
=\frac{h}{2}f(a) + \frac{h}{2}f(b) + \sum_{j=1}^n f(a+jh)
\end{split}
\end{equation}
!et
\end{verbatim}

The result becomes

\[
\begin{equation}
\begin{split}
\int_a^b f(x)dx = \sum_{j=0}^{n} \frac{1}{2} h(f(a+jh) +
f(a+(j+1)h)) \\ 
=\frac{h}{2}f(a) + \frac{h}{2}f(b) + \sum_{j=1}^n f(a+jh)
\end{split}
\end{equation}
\]

\subsubsection{Test 11: Newcommands and boldface bm vs
pmb}\label{test-11-newcommands-and-boldface-bm-vs-pmb}

First we use the plain old pmb package for bold math. The formula

\begin{verbatim}
!bt
\[ \frac{\partial\u}{\partial t} +
\u\cdot\nabla\u = \nu\nabla^2\u -
\frac{1}{\varrho}\nabla p,\]
!et
\end{verbatim}

and the inline expression
\texttt{\$\textbackslash{}nabla\textbackslash{}pmb\{u\}\ (\textbackslash{}pmb\{x\})\textbackslash{}cdot\textbackslash{}pmb\{n\}\$}
(with suitable newcommands using pmb) get rendered as

\[
 \frac{\partial\pmb{u}}{\partial t} +
\pmb{u}\cdot\nabla\pmb{u} = \nu\nabla^2\pmb{u} -
\frac{1}{\varrho}\nabla p,
\] and \(\nabla\pmb{u} (\pmb{x})\cdot\pmb{n}\).

Somewhat nicer fonts may appear with the more modern
\texttt{\textbackslash{}bm} command:

\begin{verbatim}
!bt
\[ \frac{\partial\ubm}{\partial t} +
\ubm\cdot\nabla\ubm = \nu\nabla^2\ubm -
\frac{1}{\varrho}\nabla p,\]
!et
\end{verbatim}

(backslash \texttt{ubm} is a newcommand for bold math \(u\)), for which
we get

\[
 \frac{\partial\boldsymbol{u}}{\partial t} +
\boldsymbol{u}\cdot\nabla\boldsymbol{u} = \nu\nabla^2\boldsymbol{u} -
\frac{1}{\varrho}\nabla p.
\] Moreover,

\begin{verbatim}
$\nabla\boldsymbol{u}(\boldsymbol{x})\cdot\boldsymbol{n}$
\end{verbatim}

becomes \(\nabla\boldsymbol{u}(\boldsymbol{x})\cdot\boldsymbol{n}\).

\emph{Warning.} Note: for the pandoc format, \texttt{\textbackslash{}bm}
was substituted by DocOnce to \texttt{\textbackslash{}boldsymbol}.

\subsection{Problematic equations}\label{problematic-equations}

Finally, we collect some problematic formulas in MathJax. They all work
fine in LaTeX. Most of them look fine in html too, but some fail in
sphinx, ipynb, or markdown.

\subsubsection{Colored terms in
equations}\label{colored-terms-in-equations}

The LaTeX code

\begin{verbatim}
!bt
\[ {\color{blue}\frac{\partial\u}{\partial t}} +
\nabla\cdot\nabla\u = \nu\nabla^2\u -
\frac{1}{\varrho}\nabla p,\]
!et
\end{verbatim}

results in

\[
 {\color{blue}\frac{\partial\pmb{u}}{\partial t}} +
\nabla\cdot\nabla\pmb{u} = \nu\nabla^2\pmb{u} -
\frac{1}{\varrho}\nabla p,
\]

\subsubsection{Bar over symbols}\label{bar-over-symbols}

Sometimes one must be extra careful with the LaTeX syntax to get sphinx
MathJax to render a formula correctly. Consider the combination of a bar
over a bold math symbol:

\begin{verbatim}
!bt
\[ \bar\f = f_c^{-1}\f,\]
!et
\end{verbatim}

which for pandoc output results in

\[
 \bar\boldsymbol{f} = f_c^{-1}\boldsymbol{f}.
\]

With sphinx, this formula is not rendered. However, using curly braces
for the bar,

\begin{verbatim}
!bt
\[ \bar{\f} = f_c^{-1}\f,\]
!et
\end{verbatim}

makes the output correct also for sphinx:

\[
 \bar{\boldsymbol{f}} = f_c^{-1}\boldsymbol{f},
\]

\subsubsection{Matrix formulas}\label{matrix-formulas}

Here is an \texttt{align} environment with a label and the
\texttt{pmatrix} environment for matrices and vectors in LaTeX.

\begin{verbatim}
!bt
\begin{align}
\begin{pmatrix}
G_2 + G_3 & -G_3 & -G_2 & 0 \\ 
-G_3 & G_3 + G_4 & 0 & -G_4 \\ 
-G_2 & 0 & G_1 + G_2 & 0 \\ 
0 & -G_4 & 0 & G_4
\end{pmatrix}
&=
\begin{pmatrix}
v_1 \\ 
v_2 \\ 
v_3 \\ 
v_4
\end{pmatrix}
+ \cdots
\label{mymatrixeq}\\ 
\begin{pmatrix}
C_5 + C_6 & -C_6 & 0 & 0 \\ 
-C_6 & C_6 & 0 & 0 \\ 
0 & 0 & 0 & 0 \\ 
0 & 0 & 0 & 0
\end{pmatrix}
\frac{d}{dt} &=
\begin{pmatrix}
v_1 \\ 
v_2 \\ 
v_3 \\ 
v_4
\end{pmatrix} =
\begin{pmatrix}
0 \\ 
0 \\ 
0 \\ 
-i_0
\end{pmatrix}
\end{align}
!et
\end{verbatim}

which becomes

\[
\begin{align}
\begin{pmatrix}
G_2 + G_3 & -G_3 & -G_2 & 0 \\ 
-G_3 & G_3 + G_4 & 0 & -G_4 \\ 
-G_2 & 0 & G_1 + G_2 & 0 \\ 
0 & -G_4 & 0 & G_4
\end{pmatrix}
&=
\begin{pmatrix}
v_1 \\ 
v_2 \\ 
v_3 \\ 
v_4
\end{pmatrix}
+ \cdots
\label{_mymatrixeq}\\ 
\begin{pmatrix}
C_5 + C_6 & -C_6 & 0 & 0 \\ 
-C_6 & C_6 & 0 & 0 \\ 
0 & 0 & 0 & 0 \\ 
0 & 0 & 0 & 0
\end{pmatrix}
\frac{d}{dt} &=
\begin{pmatrix}
v_1 \\ 
v_2 \\ 
v_3 \\ 
v_4
\end{pmatrix} =
\begin{pmatrix}
0 \\ 
0 \\ 
0 \\ 
-i_0
\end{pmatrix}
\end{align}
\]

The same matrices without labels in an \texttt{align*} environment:

\begin{verbatim}
!bt
\begin{align*}
\begin{pmatrix}
G_2 + G_3 & -G_3 & -G_2 & 0 \\ 
-G_3 & G_3 + G_4 & 0 & -G_4 \\ 
-G_2 & 0 & G_1 + G_2 & 0 \\ 
0 & -G_4 & 0 & G_4
\end{pmatrix}
&=
\begin{pmatrix}
v_1 \\ 
v_2 \\ 
v_3 \\ 
v_4
\end{pmatrix}
+ \cdots \\ 
\begin{pmatrix}
C_5 + C_6 & -C_6 & 0 & 0 \\ 
-C_6 & C_6 & 0 & 0 \\ 
0 & 0 & 0 & 0 \\ 
0 & 0 & 0 & 0
\end{pmatrix}
\frac{d}{dt} &=
\begin{pmatrix}
v_1 \\ 
v_2 \\ 
v_3 \\ 
v_4
\end{pmatrix} =
\begin{pmatrix}
0 \\ 
0 \\ 
0 \\ 
-i_0
\end{pmatrix}
\end{align*}
!et
\end{verbatim}

The rendered result becomes

\[
\begin{align*}
\begin{pmatrix}
G_2 + G_3 & -G_3 & -G_2 & 0 \\ 
-G_3 & G_3 + G_4 & 0 & -G_4 \\ 
-G_2 & 0 & G_1 + G_2 & 0 \\ 
0 & -G_4 & 0 & G_4
\end{pmatrix}
&=
\begin{pmatrix}
v_1 \\ 
v_2 \\ 
v_3 \\ 
v_4
\end{pmatrix}
+ \cdots \\ 
\begin{pmatrix}
C_5 + C_6 & -C_6 & 0 & 0 \\ 
-C_6 & C_6 & 0 & 0 \\ 
0 & 0 & 0 & 0 \\ 
0 & 0 & 0 & 0
\end{pmatrix}
\frac{d}{dt} &=
\begin{pmatrix}
v_1 \\ 
v_2 \\ 
v_3 \\ 
v_4
\end{pmatrix} =
\begin{pmatrix}
0 \\ 
0 \\ 
0 \\ 
-i_0
\end{pmatrix}
\end{align*}
\]

\end{document}
